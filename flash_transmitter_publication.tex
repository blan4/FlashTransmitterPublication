%%%%%%%%%%%%%%%%%%%%%%% ШАПКА %%%%%%%%%%%%%%%%%%%%%%%%%%%%
\documentclass[twoside,a4paper]{msmb} % Класс документа [twoside,a4paper]
%%%%%%%%%%%%%%%%%%%% Параметры страницы %%%%%%%%%%%%%%%%%%%
\evensidemargin 5mm \oddsidemargin 5mm \voffset -10mm \textheight 230mm \textwidth 150mm
%%%%%%%%%%%%%%%%%%%%%%%%%%%%%%%%%%%%%%%%%%%%%%%%%%%%%%%%%
\usepackage{pscyr}  % Если установлен. Устанавливается вручную, используется при подготовке макета
%%%%%%%%%%%%%%%%%%%%%%%%%%%%%%%%%%%%%%%%%%%%%%%%%%%%%%%%%%%%
%%%%%%%%%%%%  Удаляемый в сборке блок %%%%%%%%%%%%%%%%%%%%%%%
%%%%%%%%%%%%%%%%%%%%%%%%%%%%%%%%%%%%%%%%%%%%%%%%%%%%%%%%%%%%%
\begin{document}
\yearpub{2014}                                      % Год издания. Заполняет редакция.
\num{1}                                             % Номер журнала. Заполняет редакция.  
\volume{29}                                         % Сквозная нумерация, том, выпуск
%%%%%%%%%%%%%%%%%%%%%%%%%%%%%%%%%%%%%%%%%%%%%%%%%%%%%%%%%%%%%
%%%%%%%%%%%  Общие праметры для вёрстки и оформления  %%%%%%%%%%%%%%%%%
%%%%%%%%%%%%%%%%%%%%%%%%%%%%%%%%%%%%%%%%%%%%%%%%%%%%%%%%%%%%%
\renewcommand{\filename}{flash_transmitter_publication}      % Вставьте имя вашего файла. Это необходимо для отслеживания конечной страницы статьи. 
\shrtitle{Разработка мобильного приложения для передачи информации с помощью света\ldots}         % Часть названия статьи для колонтитулов.
\shrauthor{К.И.~Лейфер, И.Д.~Сиганов}  % Автор или группа авторов для оглавления и колонтитулов.
\udc{000.000}                               % Вставьте УДК.
\selectlanguage{russian}                    % Основной язык статьи 
%%%%%%%%%%%%%%%%%%%%%%%%%%%%%%%%%%%%%%%%%%%%%%%%%%%%%%%%%%%%%%%%%%%%%%%
%%%%%%%%%%%% для русскоязычной статьи и реферирования в РИНЦ  %%%%%%%%%
%%%%%%%%%%%%%%%%%%%%%%%%%%%%%%%%%%%%%%%%%%%%%%%%%%%%%%%%%%%%%%%%%%%%%%%
\title{Разработка мобильного приложения для передачи информации с помощью света} % Название статьи.
\author{К.И.~Лейфер}{kolbasisha@gmail.com} % Автор на русском.
\author{И.Д.~Сиганов}{ilya.blan4@gmail.com}
\affil{Факультет компьютерных наук, Омский государственный университет}
\abstract{В статье описана разработка приложения под Android для передачи информации с помощью света. В качестве приёмника и передатчика используется датчик света и вспышка соответственно.} % это аннотация статьи на русском языке. Постарайтесь сделать развернутую аннотацию.
\keywords{мобильные устройства, Android, java, стек протоколов, свет, передача данных}

%%%%%%%%%%%%%%%%%%%%%%%%%%%%%%%%%%%%%%%%%%%%%%%%%%%%%%%%%%%%%%%%
%%%%%%%%%%%% для статьи на английском и рефрерирования %%%%%%%%%
%%%%%%%%%%%%%%%%%%%%%%%%%%%%%%%%%%%%%%%%%%%%%%%%%%%%%%%%%%%%%%%% 
\titleEng{}     % Название статьи.
\authorEng[1]{K.I.~Leyfer}{kolbasisha@gmail.com}
\authorEng[1]{I.D.~Siganov}{ilya.blan4@gmail.com}
\affilEng[1]{Omsk State University n.a.~F.M.~Dostoevskiy}  % Организация 
\abstractEng{There is annotation.} % это аннотация статьи на ангнлийском языке
\keywordsEng{mobile devices, Android, java, protocol stack, light, data transmittion}

\maketitle % Формирование оглавления.

%%%%%%%%%%%%%%%%%%%%%% Т Е К С Т %%%%%%%%%%%%%%%%%%%%%%%%%
\section*{Введение} % Звездочка для того, чтобы раздел не нумеровался
Тут будеет введение.
%%%%%%%%%%%%%%     Литература    %%%%%%%%%%%%%%%%%%%%%%%
%%%%%%%%%%%%%% ГОСТ Р 7.0.5-2008 %%%%%%%%%%%%%%%%%%%%%%%

% можно воспользоваться системой Bibtex и стилевым файлом gost704.bst,
% входящим в состав 
\begin{thebibliography}{99}

\end{thebibliography}

\endarticle  %  Команда завершения статьи

\end{document}
